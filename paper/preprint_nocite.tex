\documentclass[11pt]{article}
\usepackage[margin=1in]{geometry}
\usepackage{amsmath,amssymb,graphicx}
\usepackage[hidelinks]{hyperref}
\title{Mechanistic Bounds and a Practical Estimator for the Solar Green Flash}
\author{%
  Jan Vorel\\
  \texttt{\href{mailto:jan.vorel@evolvion.com}{jan.vorel@evolvion.com}}
}
\date{September 3, 2025}
\begin{document}
\maketitle
\begin{abstract}
We derive a closed-form bound for the color-dependent separation of the solar upper limb near the horizon and a practical estimator for green-flash duration $\tau$ from video. Using standard-air dispersion (Edl\'en) and a horizon refraction scale $R_0\!\approx\!35.4'$ we obtain
$\Delta R \approx R_0\frac{(n(\lambda_g)\!-\!1)-(n(\lambda_r)\!-\!1)}{n(\lambda_0)\!-\!1}$, which for $\lambda_g\!=\!546.1$\,nm and $\lambda_r\!=\!650$\,nm yields $\Delta R\!\approx\!12''$. With descent rate $v_{\rm alt}\!=\!1920''/t_{\rm contact}$ the predicted duration is $\tau\!\approx\!\Delta R/v_{\rm alt}$ (typically $0.5$--$1.5$\,s). A necessary visibility condition is ${\rm FWHM}_{\rm system}\!<\!\Delta R$. We specify FlashBench, an open benchmark for field validation. Robustness scans (Ciddor vs.\ Edl\'en; density scaling) change the ratio by $\lesssim$2\%. Limitations: mirage geometry can extend $\tau$; our bound isolates dispersion. 
\end{abstract}

\section{Principles and assumptions}
Atmospheric refraction lifts the apparent solar limb; dispersion makes the lift wavelength-dependent. We assume clear horizon, standard-air dispersion, and small-angle regimes.

\section{Mechanistic bound}
Let $n(\lambda)$ be the refractive index of air. For small variations, refraction approximately scales with refractivity $(n\!-\!1)$. Taking $R_0$ as a horizon refraction scale,
\begin{equation}
\Delta R \approx R_0\frac{\Delta(n-1)}{(n(\lambda_0)-1)}\,,
\end{equation}
with $\lambda_0=589.3$\,nm. Using Edl\'en's formula for standard air gives $\Delta R\approx 12''$ between 546 and 650\,nm.

\section{Estimator and thresholds}
From a video, estimate $t_{\rm contact}$ between first and last limb contacts; $v_{\rm alt}=1920''/t_{\rm contact}$. Then $\tau_{\rm pred}=\Delta R/v_{\rm alt}$. Require pixel scale $\le \Delta R/2$ and fps $\ge 3/\tau_{\rm pred}$ for multi-frame detection.

\section{Robustness and adversaries}
Edl\'en vs.\ Ciddor dispersion alters $\Delta R$ ratio by $\lesssim 2\%$. Density scaling cancels in the ratio. Adversaries: large $v_{\rm alt}$ (low latitudes), poor sampling, and thermal-layer mirages that modify geometry and extend durations.

\section{Related work}
Green-rim thickness estimates and simulations appear in Young's analyses~\cite{YoungJOSA,YoungSim};
empirical refraction models in Bennett~\cite{Bennett1982};
dispersion formulae in Edl\'en~\cite{Edlen1966} and Ciddor~\cite{Ciddor1996}.

\section{Reproducibility}
This paper is generated by \texttt{make paper}; results by \texttt{make reproduce}. See repository root.

\section{Limitations}
The bound captures dispersion only; layered temperature inversions and mirages require ray-tracing.

\nocite{Ciddor1996,Edlen1966,Bennett1982,YoungJOSA,YoungSim}
\bibliographystyle{plain}
\bibliography{refs}
\end{document}
